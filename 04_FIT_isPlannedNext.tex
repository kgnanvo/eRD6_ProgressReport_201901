\subsubsection{Florida Tech} 
\begin{enumerate}
\item \textbf{Forward Tracker Prototype:} Our main goal for the next funding period is to extract the performance characteristics of the low-mass prototype from the data that we are hopefully about to collect at the Fermilab beam test and to present the results at conferences and in a publication. In addition, we will perform additional measurements on the detector with X-rays at Florida Tech, e.g.\ gain curves.
%
\item \textbf{EIC Simulations:} The next step in this study is to measure the track resolution of the electron track through the chambers. This will be implemented by adding a dummy plane about 15 cm past the third GEM ring. The simulated electron tracks will be focused on either the $x$ or $y$ axis, and the positions of events on the chosen axis will be histogrammed. In this fashion, one will be able to tell how the reduction of material in the GEM chambers affects the spread of position values read after the chambers due to multiple scattering. It is expected that the chromium configuration will produce a smaller spread than the standard configuration.
%
\item \textbf{$\mu$RWELL detector:} We plan to work closely with UVa on the design of a first prototype for a small cylindrical $\mu$RWELL detector. More details on this can be found in the proposal section below that describes the overall $\mu$RWELL detector R\&D that is proposed by the eRD6 consortium. Finally, we will assemble and commission the 10 $\times$ 10 cm$^2$ $\mu$RWELL prototype with zigzag-strip readout and characterise its performance using X-rays.
\end{enumerate}
