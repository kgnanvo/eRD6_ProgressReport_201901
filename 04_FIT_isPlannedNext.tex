\subsubsection{Florida Tech} 
\paragraph*{Forward Tracker Prototype:} Our main goal for the next funding period is to extract the performance characteristics of the low-mass prototype from the data that we are hopefully about to collect at the Fermilab beam test and to present the results at conferences and in a publication. In addition, we will perform additional measurements on the detector with X-rays at Florida Tech, e.g.\ gain curves.
%
\paragraph*{EIC Simulations:} Undergraduate Matt Bomberger will continue his EIC simulations to investigate the impact that material budgets in the forward and backward regions will have on the overall EIC detector performance. Our goal is to have results from a realistic simulation of the forward tracker region by May 2019.
%
\paragraph*{$\mu$RWELL detector:} We plan to work closely with UVa on the design of a first prototype for a small cylindrical $\mu$RWELL detector. More details on this can be found in the proposal section below that describes the overall $\mu$RWELL detector R\&D that is proposed by the eRD6 consortium. Finally, we will assemble and commission the 10 $\times$ 10 cm$^2$ $\mu$RWELL prototype with zigzag-strip readout and characterise its performance using X-rays.

%\paragraph*{Milestones for the next reporting period:} 
%\begin{itemize}
%\item ?
%\end{itemize}