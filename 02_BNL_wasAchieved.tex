\subsubsection{Brookhaven National Lab}

%%%%   FIGS  %%%%

 \begin{figure}[hbt]
	\centering
%		\includegraphics[width=0.9\columnwidth]{BNL_plots/Fig1.png}
		
	\caption{	\label{fig:Fig1.png}
 Microscope photos of the multi-zigzag patterns of the tested readout PCB. The right photo shows the measured gap spacing between neighboring electrodes to be 23$\mu$m.}
\end{figure}
 
%%%%%%%%%%

  \begin{figure}[hbt]
      \centering
%          \includegraphics[width=0.9\columnwidth]{BNL_plots/Fig2.png}

      \caption{      \label{fig:Fig2.png}
 Top: Microscope pictures of three different zigzag PCB’s, including a non-optimized zigzag patterned PCB manufactured using chemical etching, an optimized pattern developed using chemical etching and a recent PCB generated by Elvia using laser etching, respectively. The red band corresponds to the region on each pattern where there are no overlapping pads. Middle: corresponding cluster size distributions for the three PCB’s. Bottom: corresponding residual distributions after the small systematic global shifts in the reconstructed position (i.e., the differential nonlinearity) is removed from the data. Each distribution is fit to a double Gaussian function to reveal a dominant and background component for the residuals.}
  \end{figure}
  
%%%%%%%%%%%

  \begin{figure}[hbt]
      \centering
%          \includegraphics[width=0.9\columnwidth]{BNL_plots/Fig3.png}

      \caption{      \label{fig:Fig3.png}
 Table of design and actual PCB specifications for the three PCB’s compared above.}
  \end{figure}
  
%%%%%%%%%%%

\begin{figure}[hbt]
      \centering
%          \includegraphics[width=0.9\columnwidth]{BNL_plots/Fig4.png}

      \caption{      \label{fig:Fig4.png}
 The quadruple GEM detector (zoomed in on right picture) with multi-zigzag board readout positioned in beamline at FTBF. To the left the long silicon telescope is visible.}
  \end{figure}
  
%%%%%%%%%%%

\begin{figure}[hbt]
      \centering
%          \includegraphics[width=0.9\columnwidth]{BNL_plots/Fig5.png}

      \caption{      \label{fig:Fig5.png}
 The scatter plot on the left shows the behaviour of the position residuals across a particular zigzag pattern with the following parameters: 2mm pitch, 0.4mm zigzag period, $>$90\% interleaving, and ~1mil gap spacing. The plot on the right is the corresponding position residual distribution with a sigma equal to 53m. No attempt was made so far to subtract silicon telescope track resolution to this number. The latter contribution must be of an order of 15-20$\mu$m (in quadrature).}
  \end{figure}
  
%%%%%%%%%%%

\begin{figure}[hbt]
      \centering
%          \includegraphics[width=0.9\columnwidth]{BNL_plots/Fig6.png}

      \caption{      \label{fig:Fig6.png}
Plot showing the dependence of the spatial resolution on the degree of zigzag stretching. The zigzag parameters include a 2mm pitch, 0.4mm zigzag period, and ~1mil gap spacing. No DNL correction applied to either of the data sets.}
  \end{figure}
  
%%%%%%%%%%%

\begin{figure}[hbt]
      \centering
%          \includegraphics[width=0.9\columnwidth]{BNL_plots/Fig7.png}

      \caption{      \label{fig:Fig7.png}
 Cosmic ray stand outfitted with four GEM detector layers. The GEM’s from each detector layer are powered using an independent voltage divider (seen as the green PCB in the upper right photo), which also provides a trigger output by providing a capacitively coupled output to the bottom GEM HV. In addition, a measurement of an Fe55 spectrum is shown from one of the layers.}
  \end{figure}
  
%%%%%%%%%%%

\begin{figure}[hbt]
      \centering
%          \includegraphics[width=0.9\columnwidth]{BNL_plots/Fig8.png}

      \caption{      \label{fig:Fig8.png}
 Compact TPC engineering model and photos of the actual detector enclosure and base plate.}
  \end{figure}
  
%%%%%%%%%%%%  END FIGS  %%%%%%%%%%%%

\begin{enumerate}

\item	\textbf{\emph{Zigzag pad development}}: Over the past several months we have procured several readout PCB’s, each containing multiple versions of an optimized zigzag pattern for testing in the lab and in a beam test. The zigzag patterns were generated using a laser ablation process which has made it possible to carve out zigzag shaped electrodes with fine design parameters never tested before. In this process, a pulsed laser beam with a carefully tuned power setting is precisely translated across the PCB substrate to cut a gap between adjoining electrodes to define the zigzag shape. A gap width of about 20$\mu$m was achieved on average, well below the standard \textasciitilde75$\mu$m gap width of standard chemical etching. This in turn has allowed the interleaving (i.e., the amount by which neighboring zigzags overlap) to increase from roughly 80\% in previously tested boards to more than 90\% while maintaining the percent coverage of copper on the PCB surface at the level of 90\% or more, both of which are critical parameters in terms of the quality of charge sharing on the readout. 

Fig. \ref{fig:Fig1.png} shows zoomed in microscope photos of the zigzags on one of the PCB’s. The different zigzag geometries are arranged in an array of 100 ~1cm x ~1cm regions each filled with 2 to15 zigzag strips of a unique zigzag geometry. The gap width between adjacent strips is fixed at the minimum attainable width mentioned above for the entire board and the strip pitch ranges from 0.4 to 3.33mm, with a zigzag periodicity ranging from 330 to 1000$\mu$m. 
   
The PCB’s were generated by two different PCB manufacturers, TTM (USA), and Elvia (France) with similar fabrication results, although the Elvia boards tended to have consistently narrower gaps (\textasciitilde18$\mu$m), compared to TTM (\textasciitilde23$\mu$m). Though these boards showed a significant improvement in design reproducibility over previously manufactured boards using chemical etching, on average about 20-30\% of all strips were shorted to neighboring ones, which is something we have not experienced with chemical etching.  At the moment it is not immediately obvious how the majority of these shorts develop, however in some cases the tips of the zigzag have been seen to delaminate and fold over, bridging two electrodes to one another. We are currently working to further understand how these shorts develop and devise strategies for avoiding them. Thankfully, a substantial portion of the readout for some of the boards was unharmed and allowed a useful range of zigzag patterns to be tested. 
   
Each readout board tested was coupled to a quadruple GEM detector operated in ArCO2 (70/30) at a gain of several thousand, with 1kV/cm and 3kV/cm applied to the drift and transfer gaps respectively. Some very preliminary results from a partial x-ray scan performed in the lab are shown in Fig. \ref{fig:Fig2.png} for a PCB manufactured at Elvia and are compared to earlier results from PCB’s produced using chemical etching. In addition, the table in Fig. \ref{fig:Fig3.png} tabulates the design and actual PCB specifications of the various PCB’s that are compared.

As the overlapping region of the pattern is increased from one PCB to the next, there is a clear shift in the cluster size distribution away from single pad hits, as expected. In addition, as seen in the fit results of the residual distributions for the different PCB’s, the position resolution also improves incrementally from 93$\mu$m for the older PCB to about 63$\mu$m for the recently manufactured PCB. The background component of the newer board is believed to be due to events where the majority of the charge is collected by a single pad, which in turn deteriorates the signal to noise ratio of the peripheral pads that play a big role in regulating the centroid. Since the detector gain was particularly low for the measurement of this board, we speculate that this background component will be suppressed by increasing the gain. 

\item	\textbf{\emph{Beam Test with multi-zigzag PCB}}: We have conducted a beam test of the multi-zigzag PCB’s described above at the FTBF in March 2018. The detector was placed in the primary 120GeV proton beam at FTBF and positioned just downstream of a high precision silicon tracking telescope for the purpose of measuring the position resolution. The detector was mounted to a XY-movable stand such that the beam axis is normally incident with respect to the readout plane, as shown in Fig. \ref{fig:Fig4.png}. As the XY-table translates the detector in a plane orthogonal to the beam axis, this allows the beam to scan across each zigzag pattern, thereby allowing a multitude of zigzag geometries to be studied in a single board. The zigzag parameters for this board span an interesting range of the parameter space and were chosen for the goal of revealing important behavioral trends.

Some results from the beam test are shown below in Fig. \ref{fig:Fig5.png} for a particular set of zigzag parameters and indicate a substantial improvement over earlier zigzag boards. The beam position in the detector is reconstructed using a simple charge weighted mean of the fired strips (or centroid) using only 2,3, and 4 strip clusters. A position residual is formed with the position as determined from the high resolution silicon telescope. The scatter plot of the residuals vs the actual hit position exhibits systematic up and down displacements, which may be interpreted as deviations from linearity. This so-called differential non-linearity (DNL) is characteristic of zigzag shaped strips, however the magnitude of these deviations appears to be significantly suppressed compared to earlier zigzag PCB’s. For example, the maximum mean deviation from linearity in this case is below 50$\mu$m, whereas the maximum deviation for earlier boards was at best more than 70$\mu$m. In addition, the position resolution, given as the width of the 1D residual distribution in Fig. \ref{fig:Fig5.png} is 53$\mu$m, compared to 100$\mu$m for earlier zigzag boards and 50-60$\mu$m for boards with straight strips and a pitch of 400$\mu$m. 

It should be noted that this resolution was achieved with no effort to remove the DNL from the residual distributions along the zigzag coordinate. Additionally, the 1D residual distribution plot shows a negligible background with respect to the Gaussian fit. This is in contrast to earlier PCB’s tested and to the tests performed in the lab with a similar PCB, which has a noticeable background. The suppression of this background for the beam test data may be due to the operation of the GEM at higher gain at the beam test. It may also be that the discrete clusters of charge generated by x-rays in the lab have a smaller footprint on the readout, resulting in worse charge sharing. Finally, the beam test results exhibited no single strip hits, which can be a serious issue for non-optimized zigzag designs.
   
A systematic study was also performed to explore the dependence of the spatial resolution on the zigzag stretching parameter. This parameter basically describes how much the zigzag interleaving stretches beyond its neighbor, in terms of a percent of the pitch. A minimum is found at 5\% over-stretching (as shown in Fig. \ref{fig:Fig6.png}) and the resolution degrades quickly away from this point. This rather unexpected result requires further investigation and will be addressed in more detail in the future.

\item	\textbf{\emph{Measurements with GEM-based cosmic ray telescope}}: We have successfully completed the final assembly of the cosmic ray telescope. A photo of the cosmic ray stand showing the arrangement of the four detector layers is shown in Fig. \ref{fig:Fig7.png}.  A zoomed in photo of a single layer is also shown, which consists of a triple GEM stack coupled to a COMPASS readout plane consisting of ~10cm XY strips with a 400$\mu$m pitch. So far, each detector layer was commissioned by stress testing each GEM foil (by applying ~500V across the two electrodes in air) and acquiring signals from Fe55 from each detector, as indicated by the scope screen-shot of an Fe55 spectrum in Fig. \ref{fig:Fig7.png}. Additionally, a gas system panel was built specifically for this setup and is also shown in Fig. \ref{fig:Fig7.png}. 

\item	\textbf{\emph{GEM Studies using TPC gas mixtures in a compact TPC prototype}}: The engineering design of the compact TPC prototype has been completed and the assembly of the enclosure is nearing completion (Fig. \ref{fig:Fig8.png}). The fabrication of the base plate is complete and will shortly be mounted to the enclosure once it is finished.   

\end{enumerate}


\textbf{\emph{Publications}}:
\begin{enumerate}

\item A manuscript entitled, “Beam Test Results from a GEM-based Combination TPC-Cherenkov Detector” has been completed and will very shortly be submitted to the peer reviewed journal, IEEE Transaction on Nuclear Science. Consortium members from Stony Brook University and BNL are co-authors for this paper.

\item A manuscript entitled, “Design Studies for a TPC Readout Plane Using Zigzag Patterns with Multistage GEM detectors” has recently been published in the peer reviewed journal, IEEE Transactions on Nuclear Science \cite{8379440}. Consortium members from both Stony Brook University, FIT and BNL are co-authors for this paper.

\item A summary/abstract entitled, “Design Studies of High Resolution Readout Planes using Zigzags with GEM Detectors” presenting the results of new PCB’s produced using laser ablation was submitted for consideration at the 2018 IEEE NSS/MIC conference. 

\end{enumerate}


