\subsubsection{Brookhaven National Lab}

%%%%   FIGS  %%%%

 \begin{figure}[hbt]
	\centering
%		\includegraphics[width=0.9\columnwidth]{BNL_plots/Fig1.png}
		
	\caption{	\label{fig:Fig1.png}
 Microscope photos of the multi-zigzag patterns of the tested readout PCB. The right photo shows the measured gap spacing between neighboring electrodes to be 23$\mu$m.}
\end{figure}
 
%%%%%%%%%%

  \begin{figure}[hbt]
      \centering
%          \includegraphics[width=0.9\columnwidth]{BNL_plots/Fig2.png}

      \caption{      \label{fig:Fig2.png}
 Top: Microscope pictures of three different zigzag PCB’s, including a non-optimized zigzag patterned PCB manufactured using chemical etching, an optimized pattern developed using chemical etching and a recent PCB generated by Elvia using laser etching, respectively. The red band corresponds to the region on each pattern where there are no overlapping pads. Middle: corresponding cluster size distributions for the three PCB’s. Bottom: corresponding residual distributions after the small systematic global shifts in the reconstructed position (i.e., the differential nonlinearity) is removed from the data. Each distribution is fit to a double Gaussian function to reveal a dominant and background component for the residuals.}
  \end{figure}
  
%%%%%%%%%%%

  \begin{figure}[hbt]
      \centering
%          \includegraphics[width=0.9\columnwidth]{BNL_plots/Fig3.png}

      \caption{      \label{fig:Fig3.png}
 Table of design and actual PCB specifications for the three PCB’s compared above.}
  \end{figure}
  
%%%%%%%%%%%

\begin{figure}[hbt]
      \centering
%          \includegraphics[width=0.9\columnwidth]{BNL_plots/Fig4.png}

      \caption{      \label{fig:Fig4.png}
 The quadruple GEM detector (zoomed in on right picture) with multi-zigzag board readout positioned in beamline at FTBF. To the left the long silicon telescope is visible.}
  \end{figure}
  
%%%%%%%%%%%

\begin{figure}[hbt]
      \centering
%          \includegraphics[width=0.9\columnwidth]{BNL_plots/Fig5.png}

      \caption{      \label{fig:Fig5.png}
 The scatter plot on the left shows the behaviour of the position residuals across a particular zigzag pattern with the following parameters: 2mm pitch, 0.4mm zigzag period, $>$90\% interleaving, and ~1mil gap spacing. The plot on the right is the corresponding position residual distribution with a sigma equal to 53m. No attempt was made so far to subtract silicon telescope track resolution to this number. The latter contribution must be of an order of 15-20$\mu$m (in quadrature).}
  \end{figure}
  
%%%%%%%%%%%

\begin{figure}[hbt]
      \centering
%          \includegraphics[width=0.9\columnwidth]{BNL_plots/Fig6.png}

      \caption{      \label{fig:Fig6.png}
Plot showing the dependence of the spatial resolution on the degree of zigzag stretching. The zigzag parameters include a 2mm pitch, 0.4mm zigzag period, and ~1mil gap spacing. No DNL correction applied to either of the data sets.}
  \end{figure}
  
%%%%%%%%%%%

\begin{figure}[hbt]
      \centering
%          \includegraphics[width=0.9\columnwidth]{BNL_plots/Fig7.png}

      \caption{      \label{fig:Fig7.png}
 Cosmic ray stand outfitted with four GEM detector layers. The GEM’s from each detector layer are powered using an independent voltage divider (seen as the green PCB in the upper right photo), which also provides a trigger output by providing a capacitively coupled output to the bottom GEM HV. In addition, a measurement of an Fe55 spectrum is shown from one of the layers.}
  \end{figure}
  
%%%%%%%%%%%

\begin{figure}[hbt]
      \centering
%          \includegraphics[width=0.9\columnwidth]{BNL_plots/Fig8.png}

      \caption{      \label{fig:Fig8.png}
 Compact TPC engineering model and photos of the actual detector enclosure and base plate.}
  \end{figure}
  
%%%%%%%%%%%%  END FIGS  %%%%%%%%%%%%

\begin{enumerate}

\item	\textbf{\emph{TPC prototype}}: The assembly and  preliminary testing of the TPC prototype has been completed. The new gas enclosure has been checked for leak tightness and the 20kV feed-through passed stringent HV tests. Once the TPCC field cage, GEM's and readout PCB were all mounted to the base plate, the apparatus as a whole successfully passed HV tests up to 25kV applied to the top plate of the field cage, well above normal operating values. 

Once a cosmic trigger was established using the coincident signal from three scintillation counters, the TPC was able to be triggered and read out using the SRS/APV25 DAQ electronics to measure cosmic ray tracks at slight inclinations to the readout plane. The newly constructed cosmic ray telescope was triggered by the same signal and measured matching cosmic rays, which were found to be well correlated in space. These initial measurements validate that the TPC is working as expected and is ready to be used for the detailed set of investigations outlined earlier. 
 
\item	\textbf{\emph{Cosmic ray telescope}}: The four layers of the cosmic ray telescope have been remounted to the cosmic ray stand with HV, gas, and front end electronics all restored. Next we verified that all four layers were in working order, then were able to reconstruct cosmics using three layers, since the electronics for the fourth layer was needed for the TPC. As mentioned above, the apparatus performed well, as we were able to reconstruct matching tracks in both the telescope and the TPC. Each tacking layer is equipped with a COMPASS style X-Y readout plane coupled to a triple GEM and was shown to have a single point resolution of about 50-60$\mu$m during UVA beam test referenced earlier.   

\item	\textbf{\emph{Electronics for TPC readout}}: Though we have not yet read out the TPC with the SAMPA and DREAM electronics, both sets of electronics are currently being tested in the lab on other setups and will soon be available for use by the TPC. Both sets of electronics are working well, although there is still work to be done to integrate them into the RCDAQ DAQ system used by the TPC. 

\item	\textbf{\emph{Zigzag readout}}: As reported in earlier progress reports, we have used laser ablation to generate a variety of zigzag patterned readout strips which were measured in a beam test at Fermilab last March. The results from this beam test are now complete and are beginning to reveal the dependence of the spatial resolution on the zigzag parameters... 
Since the beam test, we have visited a PCB manufacturing facility in Coutances, France and worked directly with a laser ablation machine in an effort to optimize the working parameters, including the laser intensity and the number of laser passes over a given region of the design. The trip was extremely productive as we learned quite a lot about the subtleties of the laser ablation process, which will allow us to refine the technique for our purposes.
Examination of laser etched PCB's versus wet etched PCB's...

\end{enumerate}


\textbf{\emph{Publications}}:
\begin{enumerate}

\item A manuscript entitled, “Beam Test Results from a GEM-based Combination TPC-Cherenkov Detector” has been submitted to the peer reviewed journal, IEEE Transaction on Nuclear Science. Consortium members from Stony Brook University, Yale, and BNL are co-authors for this paper.


\item An oral presentation entitled, “Design Studies of High Resolution Readout Planes using Zigzags with GEM Detectors” discussing the results of the new zigzag PCB’s produced using laser ablation was given at the 2018 IEEE NSS/MIC conference in Sydney, Australia last November. 

\end{enumerate}


