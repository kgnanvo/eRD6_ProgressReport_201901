\subsubsection{Temple University} 
For this funding cycle TU has planned to complete the eRD3 carry over work that is listed below.
\begin{enumerate}
\item Determine shorting issue that was seen in the first two commercial triple-GEM detectors last summer.
\item Build two more commercial triple-GEM detectors using remaining GEM, HV, and readout foils, and correct the electrical shorting problem.
\item Verify the electrical shorting issue has been resolved.
\item Characterize the two newly formed triple-GEM detectors.
\end{enumerate}

In parallel to the above hardware work, we are also working on simulating a MPGD detector used in $\mu$TPC mode as a possible detector replacement/addition to a TPC in an EIC detector stack located at a second IR. Below lists several simulation goals that TU had set.

\begin{enumerate}
\item Develop machinery to simulate a $\mu$TPC detector, where dead material, gas, drift gap size, hit points, and resolution parameters can be easily adjusted.
\item Use test beam data from $\mu$TPC with COMPASS readout (taken by BNL) to implement realistic resolution parameters.
\item Develop more realistic detector digitization where the hit point resolution varies as a function of barrel radius (\emph{i.e.} first and last hit point before readout could have different resolutions).
\item Once detector is fully simulated this and the forward MPGD simulation work being done by FIT can be integrated into one cohesive tracking simulation. 
\end{enumerate}
