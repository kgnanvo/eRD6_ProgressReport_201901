\subsubsection{INFN Trieste} 
The task of the INFN participants to the eRD6 
Consortium is 
"Further development of hybrid MPGDs for single photon 
detection synergistic to TPC read-out sensors".
\par
Particle identification of electrons and hadrons
over a wide momentum range is a key ingredient 
for the physics programme at EIC. One of the most
challenging aspects is hadron identification at
high momenta, namely above 6-8~GeV/c, where the
only possibility is the use of Cherenkov imaging
techniques with gaseous radiator. The overall
constrains of the experimental set-ups at a
collider impose a limited RICH detector length 
and to operate in magnetic fringing field. 
The use, for this RICH,  
of gaseous photon detectors is 
one of the 
most likely choice. The goal of our project 
is an R\&D to further develop MPGD-based single photon
detectors in order to establish 
one of the key components of the RICH for high momentum
hadrons. This R\&D has also 
some aspects synergistic to the development 
of TPC sensors: the miniaturization of the 
read-out elements and the reduction of the 
Ions Back-Flow (IBF). 
\par
The starting point are the hybrid MPGD detectors
of single photons developed for the upgrade of 
the gaseous RICH 
counter~\cite{ALBRECHT2005215,
ABBON2008371,ABBON201021,
ABBON201126} of the COMPASS 
experiment~\cite{ABBON2007455,ABBON201569} 
at CERN SPS. These detectors are the result of 
several years of dedicated 
R\&D~\cites{ALEXEEV2009174,
ALEXEEV2010396,ALEXEEV2010129,
Alexeev:2012tba,
1748-0221-5-03-P03009,ALEXEEV2011130,ALEXEEV2012159,
1748-0221-7-02-C02014,
Alexeev:2012rva,ALEXEEV2013264,
1748-0221-8-12-C12005,
1748-0221-8-01-P01021,
ALEXEEV2014133,1748-0221-9-03-C03046,
1748-0221-9-09-C09017,
Levorato:2014gya,1748-0221-10-03-P03026,
ALEXEEV2016139,2017233}. 
They
consists in three
multiplication stages:
two THick GEMs (THGEM) layers, the first one
coated with a CsI film and acting as 
photocathode, followed by a resistive MicroMegas (MM)
multiplication stage. The COMPASS photon detectors 
can operate at gains of at least~3$\times$10$^4$
and exhibit an IBF rate lower 
than~5\%~\cite{ALEXEEV2016139,
ALEXEEV201796,1748-0221-12-07-C07026,AGARWALA2017,
AGARWALA2018}. 
An original element of
the hybrid MPGD photon detector is the approach
to a resistive MM by discrete elements:
the anode pads
facing the micromesh are
individually equipped with large-value resistors
and the HV is provided, via these resistors, to
the anode electrodes, while the micromesh is
grounded. A second set of electrodes (pads
parallel to the first ones) are embedded in the
anode PCB: the signal is transferred by 
capacitive coupling to these electrodes, 
which are connected to the front-end 
read-out electronics.
\par
The whole R\&D project develops over several years 
and it includes
further improvements of the hybrid MPGD-based photon 
detectors in order to match the requirements of high 
momenta hadron identification at EIC and initial tests
relative to  the application in gaseous detectors of a 
novel photocathode concept, based on NanoDiamond (ND) 
particles~\cite{VELARDI20171}.