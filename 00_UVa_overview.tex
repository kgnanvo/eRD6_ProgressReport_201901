\subsubsection{University of Virginia} 
The focus of the group at UVa is the development of high performance, large and low mass GEM detector  for the forward region of an EIC detector. 

Our R\&D  at UVa shares some similarities with the development by the Florida Tech group and by Temple U. group within the eRD3 program but we are specifically focused on the development of high performance, large area two dimensional U-V strip readout with fine pitch to provide excellent spatial resolution in both radial and azimuthal direction. A first prototype of such detector was built and successfully tested at the Fermilab Test Beam Facility (FTBF) in 2013. The analysis of the test beam data fully validated the expected performances of the U-V strip readout and the results were published in \cite{Gnanvo:2015xda}. 

We  recently completed the second phase of the R\&D with a design improvement of the U-V strip readout to push even further the spatial resolution capabilities in both dimensions. The new prototype is conceived around the "Common EIC GEM foil design" jointly developed by UVa, Florida Tech (FIT)  and Temple University (TU) groups of the eRD6 consortium. We have also been testing new ideas such as the ultra low mass Chromium GEM foil to reduce even further the material budget of EIC-FT GEM detectors and the development of the double-sided zebra connection scheme to provide an elegant solution for the fine pitch U-V strip readout layer. We anticipate that several of these innovative ideas will ultimately be integrated in the final design of the large area EIC  forward GEM trackers.

We are also conducting in collaboration with TU and FIT, some R\&D on a new MPGD detector technology, the $\mu$RWELL device, that we view as an alternative to GEM or Micromegas and an ideal candidate for cylindrical tracking device in the barrel region of the EIC detector. 

