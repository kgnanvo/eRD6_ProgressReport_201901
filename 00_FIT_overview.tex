\subsubsection{Florida Tech} 
The Florida Tech group has been focusing on the development of large low-mass GEM detectors with low channel count for the forward tracker (FT) of the EIC detector. In the current funding cycle the group has begun shifting focus towards R\&D on cylindrical $\boldmath{\mu}$RWELL detectors for a fast central tracker at an EIC detector.

We initially designed and implemented radial zigzag strips on large readout PCBs to achieve low-channel count while maintaining good spatial resolution. We constructed a first one-meter-long prototype with such a readout at Florida Tech using a purely mechanical construction technique without any gluing and tested it in beams at Fermilab in 2013. This study showed a non-linearity in the position measurement of hits\cite{Zhang:2015pqa}. The reason was an over-etching of tips and under-etching of troughs in the zigzag strips, which caused insufficient interleaving of adjacent strips and consequently insufficient charge sharing among strips. We adjusted the zigzag strip design to improve the strip interleaving. Small PCBs and a flex-foil with the improved zigzag strip design were produced by industry and by CERN, respectively. We subsequently tested these with highly collimated X-rays at BNL. A substantial reduction in the non-linearity and an improvement in spatial resolution were observed\cite{Zhang:2017dqw}.

Next, we designed a second large Triple-GEM detector that implements the drift electrode and a readout electrode with improved radial zigzag strips on polyimide foils rather than on PCBs to reduce the material in the active detector area\cite{Hohlmann:2017sqj}. These foils were then produced by CERN. To provide sufficient rigidity to this new detector while maintaining low mass, we produced the main support frames from carbon fiber material. We designed the GEM foils for this second detector in such a way that they can also be used for the second UVa FT prototype (``common GEM foil design''). A number of these GEM foil were produced for Florida Tech and UVa by the CERN workshop using the single-mask foil etching technique. Assembly of this second prototype showed that 3D-printed pull-out and frame components made from ABS did not have sufficient material strength to sustain the mechanical forces needed for stretching. They are now being replaced by stronger components made from PEEK.