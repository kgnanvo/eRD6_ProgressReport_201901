\subsubsection{University of Virginia} 
The main goals for the current cycle were:

%
\begin{enumerate}[leftmargin=*]
\item \textbf{Large EIC-FT-GEM prototype:} Start the analysis of the FTBF2018 test beam  data. Continue the characterization of the prototype with cosmic and x-ray at UVa and at the BNL x-ray scan setup if required.  Start preparing the draft paper for publication of these results in peer-reviewed journal. 
%
\item \textbf{R\&D on $\mu$RWELL detector technology:} Work closely with FIT on the design of a small cylindrical  $\mu$RWELL detector. We will also continue the study and characterization of our current small prototype and we are requesting additional fund to acquire additional small  10 cm $\times$ 10 cm prototypes with the R\&D focus on low mass and high resolution 2D readout strips patterns.
%
\item \textbf{VMM readout Electronics:} Acquire a small size VMM-based Scalable Readout System (SRS) and test this new promising electronics with the our large EIC-FT-GEM and  $\mu$RWELL prototypes. We will compare the performances of VMM-SRS readout system with the APV25 electronics.
%
\item \textbf{Draft paper on Chromium GEM (Cr-GEM) studies:} We plan to continue the performance study of Cr-GEMs with our existing prototype and continue drafting a paper on the results of these studies for publication in NIMA or TNS journal.
\end{enumerate}

%Below was the milestone regarding the construction and characterization in test beam of UVa large forward tracker GEM prototype.
%\paragraph*{2018 milestones:} 
%\begin{itemize}
%\item March - April 2018: Assembly of EIC-FT GEM prototype
%\item May - June 2018: Performance tests with Cosmic, X-rays, Sr$^{90}$ sources
%\item July 2018: Beam test at Fermilab
%\end{itemize}