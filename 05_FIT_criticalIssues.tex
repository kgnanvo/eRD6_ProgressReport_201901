\subsubsection{Florida Tech} 

\paragraph*{Technical Issues:} The original 3D-printed design of the pull-out posts for the low-mass detector proved to be not robust enough. Under tension force in the open detector, they tend to bend more than we expected and some show cracks. The original pull-out design is a copy of the pull-out used in the CMS forward muon GEM upgrade, but there the pull-outs are made from stainless steel whereas we attempted to use very light ABS material in the EIC prototype. We have changed the pull-out design to a solid block (Fig.~\ref{fig:Pullouts} left) that appears to be more stable. Some of the original pull-outs could be replaced with redesigned block pull-outs that are more robust before the beam test. At the time of the writing this report, the chamber is being retrofitted with additional pull-outs that are being 3D-printed at Fermilab. After the beam test, we plan to replace all pull-outs with new pull-outs made from aluminum or possibly PEEK.

\begin{figure}
	\centering
		\includegraphics[width=8.2cm, height=6cm]{FIT_plots/Pullouts.JPG}
		\includegraphics[width=8.2cm, height=6cm]{FIT_plots/Innerframes.JPG}
	\caption{\label{fig:Pullouts}Left: Comparison of original pull-out post (left) and redesigned pull-out post (right) that is more robust. Right: Gaps between inner frames on long sides (on right) cause some foil warping.}	
\end{figure}

On the long side of the detector, there are centimeter-wide gaps between the inner frames that cause some foil warping in those gaps (Fig.~\ref{fig:Pullouts} right). This in turn causes shorts between foils or HV instabilities as the distances between foils are not sufficiently constant. By contrast, the inner frames at the wide end of the trapezoid are very close to each other which keeps the foils smooth in that region. The gaps are the consequence of our original attempt to simplify the frame design by keeping all inner frame pieces at the same length. At the time of writing this report, we are retrofitting the chamber with longer inner frames on the sides that are being 3D-printed at Fermilab to close the gaps. Here we are taking advantage of the purely mechanical chamber construction method that allows a re-opening of the chamber.

\paragraph*{Manpower Issues:} The departure of our post-doc Aiwu Zhang back in December 2016 has severely slowed down progress. While our undergraduates are doing a great job and are very enthusiastic about building and testing prototype detectors and analyzing beam test data, they have limited availability and experience.