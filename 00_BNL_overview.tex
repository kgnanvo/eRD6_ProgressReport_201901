\subsubsection{Brookhaven National Lab} 
The group at BNL is mainly engaged in optimizing micro-pattern gaseous detectors (MPGD’s) for reading out a time projection chamber (TPC) for use at the EIC. 

Over the last few years we have built and tested numerous planar GEM detectors with long (\textasciitilde16mm) and short (\textasciitilde3mm) drift regions and have equipped them with both zigzag pad and strip readout geometries in an effort to study the spatial and angular resolution of a host of detector configurations. Following detailed studies of these detectors in the lab, beam tests were also carried out in 2012 and 2104 at Fermilab to fully characterize the performance of GEM detectors with extended drift gaps under beam conditions. The results of these efforts were published in a peer reviewed journal in 2014\cite{7497629}.

In addition, in collaboration with Stony Brook U. and Yale U., we have built a prototype combination TPC-Cherenkov (TPCC) detector to study the feasibility of performing tracking and pID measurements in a common detector volume. The detector was filled with a specially chosen gas to be used as both the Cherenkov radiator and the TPC working gas. After investigating important characteristics such as the drift velocity and the charge spread in various candidate gases, a beam test was conducted to demonstrate a proof of principle of the viability of this detector concept. The results from these tests were positive and are detailed in a manuscript recently submitted for publication in a peer reviewed journal (IEEE TNS). (Preliminary results from the TPCC have also already been presented at several conferences and have appeared in various conference proceedings\cite{Woody:2015ola}.)

More recently we have focused on optimizing the design of the readout plane for a GEM detector made of zigzag shaped charge collecting anodes. We initially performed simulations to study the zigzag geometry, followed by a systematic set of measurements in the lab to reveal which geometrical parameters drive the performance of the readout. The results of these investigations were recently published in a peer review journal \cite{8379440}, with collaborators from Florida Tech and Stony Brook U. The use of zigzag shaped anodes has been validated to a point that this R\&D is considered complete for the eRD6 program. However, as part of a BNL funded LDRD program, we continued to refine the design of the zigzag geometry by pushing the design parameters beyond what could be produced using standard chemical etching processes. A novel laser etching technique was used to generate PCB’s with zigzag pad geometries with significantly finer features that more closely resemble idealized patterns determined by simulation. These new PCB’s were tested in the lab and very recently at a beam test at FNAL. A summary of these investigations was recently submitted to the 2018 IEEE NSS conference for consideration. 

Currently our focus is on investigating various avalanche technologies and anode geometries for a TPC readout, including the use of various zigzag readouts with GEM’s, Micromegas, a combinations of the two, and $\mu$RWELL. We have built a prototype TPC with zigzag readout and plan to employ our newly built cosmic ray telescope to study the reconstruction of particle tracks in the lab. 

