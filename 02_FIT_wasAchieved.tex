\subsubsection{Florida Tech} 

\paragraph*{Construction and commissioning of low-mass EIC Forward Tracker GEM detector prototype:}

We assembled the new low-mass FT GEM detector with zigzag readout strips. Fig.~\ref{fig:Low-mass detector assembly} shows a couple of steps from the stack assembly process. Fig.~\ref{fig:Before and after stretching} shows the stack of five foils before and after stretching.  A new flex-circuit foil that we designed was produced by the CERN workshop and is used in this assembly. Spring-loaded pins are soldered to this HV foil and make contact with the drift foil and GEM foils to provide the appropriate electric potentials. A standard ceramic HV divider circuit from CERN provides the potentials from a single HV input voltage and is also soldered onto the HV foil (Fig.~\ref{fig:HV_foil}). After assembly, each GEM foil showed an impedance across its two faces in excess of 2 G$\Omega$ in air with 50\% humidity. After stretching the foil stack and closing the chamber with the Al-Kapton window, no deformation of the carbon fiber structure is observed, which proves its ability to take up the tension of the five stretched foils as designed. The assembled detector has a mass of about 3 kg including an on-board HV filter and cable, but not including readout electronics. This is to be compared with, e.g. the GE1/1 Triple-GEM detector for the CMS forward muon upgrade that has a mass of about 20 kg.

\begin{figure}
	\centering
%		\includegraphics[width=8.2cm, height=6cm]{FIT_plots/PlacingReadoutInStack.JPG}
%		\includegraphics[width=8.2cm, height=6cm]{FIT_plots/Stretching_1.jpg}
	\caption{\label{fig:Low-mass detector assembly}Assembly of low-mass forward tracker prototype at Florida Tech. Left: Placement of readout foil at bottom of foil stack. Right: Stretching of completed foil stack with electronic torque screwdriver.}	
\end{figure}

\begin{figure}
	\centering
%		\includegraphics[width=8.2cm]{FIT_plots/StackBeforeStretching.jpg}
%		\includegraphics[width=8.2cm]{FIT_plots/StretchedStackBeforeClosing.jpg}
	\caption{\label{fig:Before and after stretching}Stack of readout foil, three GEM foils, and drift foil before (left) and after (right) stretching against 3D-printed pull-out posts mounted on carbon fiber frame. Note the absence of spacers in the active area.}	
\end{figure}

\begin{figure}
	\centering
%		\includegraphics[width=8.2cm, height=6cm]{FIT_plots/HVfoil_1.jpg}
%		\includegraphics[width=8.2cm, height=6cm]{FIT_plots/HVfoil_on_CFframe.JPG}
	\caption{\label{fig:HV_foil}High-voltage foil for spring-loaded pins that make contact with the drift foil and GEM foils to provide the appropriate electric potentials. Left: Alignment check against GEM foil. Right: HV foil glued to carbon fiber frame with HV pins and ceramic HV divider soldered onto HV foil.}	
\end{figure}

The HV stability and linearity of the HV divider was successfully tested in pure CO$_2$ up to a drift voltage of -4600V. Commissioning in Ar/CO$_2$ 70:30 is ongoing. At the time of the writing of this report, the detector is being tested at the Fermilab test beam facility. This test was prepared in close coordination with the UVa and BNL groups as planned. UVa and Florida Tech are sharing one setup; BNL is providing a reference tracker composed of four small GEMs for this setup.

\paragraph*{R\&D on $\boldmath{\mu}$RWELL detector:} We have ordered a 10$\times$10 cm$^2$ resistive micro-well detector from CERN to begin some basic R\&D on this detector technology for fast tracking in the barrel region of the EIC detector. To complement the 2D readout with Cartesian strips chosen by the UVa group for their $\mu$RWELL detector prototype, we opted for a 1D zigzag strip readout foil based on the foil design that we had used for the 10$\times$10 cm$^2$ prototype\cite{Zhang:2017dqw} of the low-mass FT detector. We expect to receive this small detector in August or September 2018.

\paragraph*{EIC Simulations:} Undergraduate student Matt Bomberger began work on EIC simulations for investigating the impact that material budgets in the forward and backward regions will have on the overall EIC detector performance. He installed the EICroot simulation framework on our computers. Using a basic example, he learned how to run the simulation, change detector parameters and plot their impact on the momentum resolution of forward particles. Matt will go to BNL in early August for a week to work directly with EICRoot expert Alexander Kiselev on the implementation of a forward tracker based on Triple-GEMs.

