\subsubsection{Temple University} 
Temple University (TU) has been focusing mainly on completing the the assembly and characterization of the commercial triple-GEM detectors. This R\&D work was carried over from the eRD3 + eRD6 merger. These commercial triple-GEM detectors follow the STAR FGT triple-GEM design~\cite{STARfgt}, but use commercial GEM, HV, and readout foils that were produced by Tech-Etch. Additionally these detectors are investigating the use of Kapton spacer rings in place of the more traditional G10 spacer grids. We were able to identify the cause of the electrical short that led to excessive sparking in the first two commercial triple-GEM detectors that were built last summer. Using the remaining materials, we are able to build two more commercial triple-GEM detectors prototypes 3 and 4. The third detector has now been completely assembled and is undergoing characterization, while the all foils for the fourth commercial triple-GEM have been stretched and glued to their respective frames. During the third and fourth detector assemblies we implemented a fix to prevent the same shorting that occurred in the first two triple-GEM detectors. This fix was verified with the third chamber that we built.

 The EIC is likely to contain two detector stacks, each at a different IR, and it would be beneficial to not have two identical detector stacks. In an effort to find an alternative technology to a TPC, which would most likely be contained in one of the detector stacks, we are beginning to focus on some simulation work to quantify alternate technology choices. In particular the use of an MPGD in $\mu$TPC mode as a replacement or in addition to a TPC. This simulation work is being done to evaluate the performance of such a detector in a second EIC detector. We now have the basic machinery in place which enables us to easily adjust the dead material, gas, drift gap size, number of hit points, and specify the hit point, radial, and transverse resolutions of the detector. Work has now just started to modify these parameters to simulate a realistic detector digitization. To determine the proper resolution parameters test beam data taken by BNL with a MPGD operating in $\mu$TPC mode using a COMPASS styled readout will be used. Ultimately this simulation work of the $\mu$TPC will be integrated into the simulation work that is being done at FIT relating to the forward MPGD tracking. This integration will allow for a full tracking simulation which covers the mid-rapidity region consisting of a vertex detector and a MPGD $\mu$TPC barrel detector, as well as the forward/backward regions consisting of tracking MPGD detectors.