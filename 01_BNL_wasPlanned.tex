\subsubsection{Brookhaven National Lab}

\begin{enumerate}

\item	\textbf{\emph{TPC prototype}}: We planned to complete the assembly and commissioning of our new TPC prototype detector, which reuses the 10cm x 10cm x 10cm field cage from our older TPCC prototype detector. Ultimately, we plan to use this prototype to study various readout plane geometries (including zigzag shaped charge collecting anodes) as well as different avalanche technologies (including GEM, Micromegas, and micro-RWELL) in a TPC application. Eventually we also plan to investigate the optimum gas mixture to be used with a particular readout pad geometry and avalanche scheme. In addition, we eventually plan to optimize the operating parameters (like field and voltage configurations) of interesting avalanche options with regard to important detector characteristics such as charge spread due to diffusion, attachment, and ion back flow.   

\item \textbf{\emph{Cosmic ray telescope}}: We last reported that once the assembly and preliminary testing of our 4-layer, GEM-based cosmic ray telescope was complete, all four layers were dismounted and immediately shipped out to Fermilab to be used by our colleagues at UVA for their beam test. Once at Fermilab, the 4 layers were successfully used as a reference tracker for other particle trackers under test, which essentially served as the commissioning run for this device. (Details of the UVA beam test may be found in later sections of this report.) For this funding cycle, we planned to re-mount the 4-layers into the cosmic ray stand and measure cosmic rays in the lab in conjunction with the newly built prototype TPC, which will also be placed in the cosmic ray stand, as the detector under test. At this early stage we did not anticipate doing any detailed studies of the TPC other than verifying that the the telescope and TPC prototype find matching tracks. 

\item	\textbf{\emph{Electronics for TPC readout}}: We planned to read out the prototype TPC with different electronics in order to compare the performance of the TPC in each case. Initially we planned to use our work-horse electronics: the SRS/APV25 system, which is readily available, but not exactly ideal for a TPC application. In addition, we also planned to read out the TPC with the SAMPA and DREAM electronics, primarily because we now have access to both technologies, but more importantly because both are directly suitable for a TPC application.  

\item	\textbf{\emph{Zigzag readout}}: As mentioned above, work on this R\&D topic is no longer funded under eRD6, however this work continues as part of a BNL funded LDRD program. We briefly report on some of this work since it has a direct impact on some aspects of the TPC R\&D done through eRD6. In short, we are continuing the development and optimization of zigzag pad geometries of readout planes for GEM, Micromeags, and micro-RWELL detectors. In particular, we are pursuing zigzag designs with unprecedented feature sizes which are generated using a novel laser ablation technique. At this stage of the research, we are focused on examining the microscopic features of the fabricated zigzag structure, including the gap width between neighboring zigzag electrodes and the gap trench depth and the overall topology. Ultimately we hope to discover how such microscopic features influence charge sharing and the overall detector performance. 



#\end{enumerate}