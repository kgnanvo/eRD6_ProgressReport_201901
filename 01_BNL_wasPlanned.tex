\subsubsection{Brookhaven National Lab} 

\begin{enumerate}

\item	\textbf{\emph{Zigzag pad development}}: We planned to continue the development and optimization of zigzag pad geometries of readout planes for GEM detectors. In particular, we planned to design zigzag pads with greater pad overlap and smaller pad-to-pad gaps, which according to our simulation results should exhibit improved performance over our older designs. We then intended to use these designs to fabricate new zigzag PCB’s by employing a laser ablation process capable of generating very high precision electrode structures on the PCB substrate. Following the production of the PCB’s we planned to measure the position resolution of several variants of the optimized zigzag geometry in the lab using our x-ray scanner. 

\item \textbf{\emph{Beam Test with multi-zigzag PCB}}: We planned to test a planar GEM detector equipped with the same new readout PCB’s comprising an array of different zigzag patterns at the Fermilab test beam facility (FTBF). As each set of pads with a unique zigzag geometry is exposed to the primary 120GeV proton beam at the FTBF, we hoped to gauge the performance of each pattern with the goal of ultimately identifying the optimum zigzag parameters for a particular GEM detector configuration.  

\item	\textbf{\emph{Measurements with GEM-based cosmic ray telescope}}: We planned to complete the assembly and commissioning of a GEM based cosmic ray telescope. The telescope consists of four triple GEM detector layers, each outfitted with a COMPASS style readout board consisting of XY strips. The telescope should be capable of reconstructing reference particle tracks to relatively high precision such that it may be used to obtain a realistic measure of the position resolution for most of the detectors we plan to study within a lab setting. This is in contrast to the position resolution measured with our x-ray scanner, which only allows a measure of the resolution using discrete clusters of charge produced by x-ray conversions within the detector gas.   

\item	\textbf{\emph{GEM Studies using TPC gas mixtures in a compact TPC prototype}}: We planned to complete the assembly of a new compact TPC enclosure to house a pre-existing 10cm x 10cm x 10cm field cage used for the TPCC studies mentioned above. The new TPC will be read out with either one of our optimized zigzag PCB’s or another suitable readout plane, and will utilize a quadruple GEM stack, or Micromegas detector, or some combination of the two. Each detector variant will be tested to find an optimal readout configuration for maximizing detector performance. In addition, the TPC may be used in conjunction with the cosmic ray telescope to study the reconstruction of particle tracks using different gases and operating parameters, including studying charge spread due to diffusion, space charge effects, attachment in the gas, among other gas characteristics. 


\end{enumerate}